\documentclass [oneside, letterpaper] {article}

\usepackage {microtype}

\title {Password Manager}
\author {Theodore Contes \and A. Connor Waters}
\date{}

\begin {document}
    \maketitle

    \section {Introduction}
        Good passwords are a critical part of keeping information secure in the
        current day, but maintaining good passwords is becoming more difficult
        every day. Not only do you need long, complex passwords to defeat
        standard attacks, such as dictionary attacks or brute force methods, but
        you need multiple such passwords. Password reuse leaves oneself open to
        having multiple accounts and services vulnerable if any password is
        compromised. But the alternative is to memorize many passwords, and then
        memorize new ones when you need to change those passwords. For many,
        this is not a practical solution, and so they instead leave themselves
        open to attacks through weak passwords and password reuse. 

        This project aims to provide a solution to the problem, by allowing a
        user to memorize a single strong password, and then store the rest of
        their password information behind this password. The password manager
        will create and store passwords in encrypted files, allowing the user to
        access them but not others.

        Password managers do have some weaknesses; inherently they rely on a
        single password to protect all of the other passwords, thus introducing
        a single point of failure. If the single password is compromised then
        all of the passwords and accounts stored within the manager are at risk.
        Alternatively, if the access to the password manager is lost, then the
        user has just lost access to all of their accounts. This can be
        mitigated through backups and storing the information and encryption
        keys on multiple computers, though this is not perfect. Finally,
        password managers cannot be used for all passwords, if only because they
        are stored on computers: to access the password manager you must first
        be able to access the computer it is on. Thus, while they greatly reduce
        the number of passwords needed, users are still required to know more
        than one strong password for good protection.

    \section {Related Work}
        The concept of a password manager is not new. There are many such
        programs offered, and they come with a vast array of useful features if
        you are willing to pay the price. As many of these services offer data
        backup, these costs are generally subscription based rather than one
        time. Interfaces to password managers range from simple terminal
        commands to browser extensions that automatically save and input
        passwords for web accounts without any prompting from the user. Our
        application uses a simple graphical user interface. The user is first
        prompted to input a master password or, if it is their first time using
        the software, prompted to create a master password, and this password is
        used to generate an encryption key to protect the password database.

    \section {Design}


    \section {Project Description}

\end {document}
